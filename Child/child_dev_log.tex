
\documentclass[12pt]{amsart}
\usepackage{geometry} % see geometry.pdf on how to lay out the page. There's lots.
%\geometry{a4paper} % or letter or a5paper or ... etc
% \geometry{landscape} % rotated page geometry

% See the ``Article customise'' template for come common customisations

\title{CHILD Development Log}
\author{Greg Tucker}
%\date{} % delete this line to display the current date

%%% BEGIN DOCUMENT
\begin{document}

\maketitle
%\tableofcontents

This is a log of development work on CHILD, beginning February 2011.

\section{14 Feb 2011}
\subsection{Stratigraphy Manager}

We need a more efficient way to input and/or modify stratigraphy information in CHILD. To this end, I am designing a new utility class called ChildStratigraphyManager (or something like that). Because stratigraphy management is something shared by a number of models (like SedFlux), I would like to make the high-level design fairly generic. So before I write any code, here are some thoughts on capabilities and design.

Capabilities of a Stratigraphy Manager should include:
\begin{enumerate}
  \item Read stratigraphy data from file (for a particular model/format)
  \item Write stratigraphy data to file
  \item Create a new stratigraphy with a given set of properties
  \item Get or set any property $p$ at any given element $i$
  \item Get or create a list of properties associated with a stratigraphy
  \item Get or set properties
\end{enumerate}

For the ``strain softening'' project with Phaedra and Peter Koons, I specifically want to be able to modify the rock erodibility parameter to a set of arbitrary values at all ``rock'' layers at all nodes, tied to the node.

I'm not sure I really need to re-engineer CHILD to make this happen. If I gave CHILD the option at startup of reading an existing stratigraphy file, regardless of whether it is a restart run or not, and with a user-specified name, then I could modify any given strat file in matlab.
The algorithm would be something like this:
\begin{enumerate}
  \item Run CHILD with a default stratigraphy, which is output
  \item Run FLAC and compute cohesion
  \item In matlab coupler script, read cohesion and convert it to erodibility
  \item In matlab, call function to read stratigraphy file
  \item In matlab, call function to modify stratigraphy file according to erodibility
  \item In matlab, call function to write strat file
  \item In matlab, modify child input file to read the modified strat file on startup or restart
\end{enumerate}

That said, it would be cleaner if CHILD's stratigraphy were managed by a quasi-independent module, as opposed to being set within tMesh. It would be nice if CHILD could read a stratigraphy file from scratch, for example. 

Ok, here's what to do. In tMesh, layers appear via ``MakeLayersFromInputData.'' This is only called from one spot within tMesh.cpp, which could be removed and replaced with an ``external'' call to a stratigraphy manager. The stratigraphy manager uses the code from that same function (moved to a new file) to read and construct layer data.

First, though, I need a regression battery to be on the safe side!


\end{document}